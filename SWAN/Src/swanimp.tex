\documentclass[12pt]{book}
\usepackage{html,a4wide}
\usepackage[british]{babel}
\newcommand{\hl}[1]{\htmladdnormallink{{\it #1}}{#1}}
\begin{document}
\pagenumbering{roman}
\pagestyle{empty}

\begin{center}
{\Huge\bf SWAN}
\end{center}
\vspace{2cm}
\begin{center}
{\Large\bf IMPLEMENTATION MANUAL}
\end{center}
\vfill
\begin{center}
{\Large\bf SWAN Cycle III version 40.72ABCDE}
\end{center}

\cleardoublepage

\noindent
{\Large\bf SWAN IMPLEMENTATION MANUAL}

\vfill

\begin{table}[htb]
\begin{tabular}{lcl}
by           &:& The SWAN team \\
             & & \\
mail address &:& Delft University of Technology \\
             & & Faculty of Civil Engineering and Geosciences \\
             & & Environmental Fluid Mechanics Section \\
             & & P.O. Box 5048 \\
             & & 2600 GA Delft \\
             & & The Netherlands \\
             & & \\
e-mail       &:& swan-info-citg@tudelft.nl \\
home page    &:& \hl{http://www.swan.tudelft.nl}
\end{tabular}
\end{table}

\vfill

\noindent
Copyright (c) 2009 Delft University of Technology.
\\[2ex]
\noindent
Permission is granted to copy, distribute and/or modify this document
under the terms of the GNU Free Documentation License, Version 1.2
or any later version published by the Free Software Foundation;
with no Invariant Sections, no Front-Cover Texts, and no Back-Cover
Texts. A copy of the license is available at
\hl{http://www.gnu.org/licenses/fdl.html\#TOC1}.

\clearpage
\pagestyle{myheadings}
\newcommand{\chap}[1] % Re-define the chapter command
       {
        \chapter{#1}
        \markboth{\hfill Chapter \thechapter \hfill}{\hfill {#1} \hfill}
       }

\tableofcontents

\chap{Introduction} \label{ch:intro}
\pagenumbering{arabic}

This Implementation Manual is a part of the total material to implement the
SWAN wave model on your computer system. The total material consists of:
\begin{itemize}
    \item the SWAN source code,
    \item the SWAN executable for Microsoft Windows,
    \item the User Manual,
    \item this Implementation Manual,
    \item the Technical documentation,
    \item the SWAN programming rules,
    \item utilities and
    \item some test cases.
\end{itemize}
All of the material can be found on the following SWAN web page\\
\hl{http://www.swan.tudelft.nl}.
\\[2ex]
\noindent
On the SWAN home page \hl{http://www.swan.tudelft.nl},
general information is given about the functionalities, physics and limitations
of SWAN. Moreover, the modification history of SWAN is given. Finally,
information on support, links to the related web pages and various
free software are provided.
\\[2ex]
\noindent
After downloading the material, you may choose between
\begin{itemize}
  \item direct usage of the SWAN executable for Windows and
  \item implementation of SWAN on your computer system.
\end{itemize}
If you want to use the SWAN executable available on the SWAN web site, please read
Chapters \ref{ch:usage} and \ref{ch:test} for further information.
\\[2ex]
\noindent
For the purpose of implementation, you have access to the source code of SWAN and
additional files, e.g. for testing SWAN. Please read the copyright in this
manual and in the source code with respect to the terms of usage and distribution
of SWAN. You are permitted to implement SWAN on your computer system.
However, for any use of the SWAN source code in your environment, proper reference
must be made to the origin of the software!
\\[2ex]
\noindent
Implementation involves the following steps:
\begin{enumerate}
   \item Copying the source code from the SWAN web page to the computer system
         on which you want to run SWAN.
   \item If necessary, applying patches for an upgrade of the source code due to
         e.g., bug fixes, new features, etc.
   \item Making a few adaptions in installation-dependent parts of the code.
   \item Compiling and linking the source code to produce an executable of SWAN.
   \item Testing of the executable SWAN.
\end{enumerate}
After the last step you should have the executable SWAN ready for usage. Note
that steps 3 and 4 can be done fully automatically.

\section{The material}

The downloaded file {\tt swan4072.tgz} contains the SWAN source code. You can unzip this
file either with WinZip (in case of Windows) or with the command {\tt tar xzf} (in case of
UNIX or Linux). The SWAN source code consists of the following files:
\begin{tabbing}
xxxxxxxxxxxxxxxxxxxxxxxx\= \kill
main program             \>:       swanmain.ftn \\
pre-processing routines  \>:       swanpre1.ftn \\
                         \> $\,\,$ swanpre2.ftn \\
computational routines   \>:       swancom1.ftn \\
                         \> $\,\,$ swancom2.ftn \\
                         \> $\,\,$ swancom3.ftn \\
                         \> $\,\,$ swancom4.ftn \\
                         \> $\,\,$ swancom5.ftn \\
post-processing routines \>:       swanout1.ftn \\
                         \> $\,\,$ swanout2.ftn \\
service routines         \>:       swanser.ftn \\
                         \> $\,\,$ SwanIntgratSpc.ftn90 \\
\end{tabbing}
\newpage
\begin{tabbing}
xxxxxxxxxxxxxxxxxxxxxxxx\= \kill
routines for support     \> \\
parallel MPI runs        \>:       swanparll.ftn \\
routines for unstructured \> $\,\,$ \\
grids                     \>:       SwanReadGrid.ftn90 \\
                          \> $\,\,$ SwanReadADCGrid.ftn90 \\
                          \> $\,\,$ SwanReadTriangleGrid.ftn90 \\
                          \> $\,\,$ SwanReadEasymeshGrid.ftn90 \\
                          \> $\,\,$ SwanInitCompGrid.ftn90 \\
                          \> $\,\,$ SwanCheckGrid.ftn90 \\
                          \> $\,\,$ SwanCreateEdges.ftn90 \\
                          \> $\,\,$ SwanGridTopology.ftn90 \\
                          \> $\,\,$ SwanGridVert.ftn90 \\
                          \> $\,\,$ SwanGridCell.ftn90 \\
                          \> $\,\,$ SwanGridFace.ftn90 \\
                          \> $\,\,$ SwanPrintGridInfo.ftn90 \\
                          \> $\,\,$ SwanFindPoint.ftn90 \\
                          \> $\,\,$ SwanPointinMesh.ftn90 \\
                          \> $\,\,$ SwanBpntlist.ftn90 \\
                          \> $\,\,$ SwanPrepComp.ftn90 \\
                          \> $\,\,$ SwanVertlist.ftn90 \\
                          \> $\,\,$ SwanCompUnstruc.ftn90 \\
                          \> $\,\,$ SwanDispParm.ftn90 \\
                          \> $\,\,$ SwanPropvelX.ftn90 \\
                          \> $\,\,$ SwanSweepSel.ftn90 \\
                          \> $\,\,$ SwanPropvelS.ftn90 \\
                          \> $\,\,$ SwanTranspAc.ftn90 \\
                          \> $\,\,$ SwanTranspX.ftn90 \\
                          \> $\,\,$ SwanDiffPar.ftn90 \\
                          \> $\,\,$ SwanGSECorr.ftn90 \\
                          \> $\,\,$ SwanInterpolatePoint.ftn90 \\
                          \> $\,\,$ SwanInterpolateAc.ftn90 \\
                          \> $\,\,$ SwanInterpolateOutput.ftn90 \\
                          \> $\,\,$ SwanConvAccur.ftn90 \\
                          \> $\,\,$ SwanConvStopc.ftn90 \\
                          \> $\,\,$ SwanThreadBounds.ftn90 \\
                          \> $\,\,$ SwanFindObstacles.ftn90 \\
                          \> $\,\,$ SwanCrossObstacle.ftn90 \\
                          \> $\,\,$ SwanComputeForce.ftn90 \\
routines for installation \>:       ocpids.ftn \\
command reading routines  \>:       ocpcre.ftn \\
miscellaneous routines    \>:       ocpmix.ftn \\
\end{tabbing}
\newpage
\begin{tabbing}
xxxxxxxxxxxxxxxxxxxxxxxx\= \kill
general modules           \>:       swmod1.ftn \\
                          \> $\,\,$ swmod2.ftn \\
modules for XNL           \>:       m\_constants.ftn90 \\
                          \> $\,\,$ m\_fileio.ftn90 \\
                          \> $\,\,$ serv\_xnl4v5.ftn90 \\
                          \> $\,\,$ mod\_xnl4v5.ftn90 \\
modules for unstructured  \> $\,\,$ \\
grids                     \>:       SwanGriddata.ftn90 \\
                          \> $\,\,$ SwanGridobjects.ftn90 \\
                          \> $\,\,$ SwanCompdata.ftn90 \\
\end{tabbing}

\noindent
The source code is written in Fortran 90. Some routines are written in fixed form and depending on your system,
the extension may be {\tt for} or {\tt f}. Other routines are written in free form and are indicated by
extension {\tt f90}. The conversion from {\tt ftn} or {\tt ftn90} to one of these extensions
can be done automatically or manually; see Chapter \ref{ch:instal}.
\\[2ex]
\noindent
You are allow to make changes in the source code of SWAN, but Delft University of Technology
will not support modified versions of SWAN. If you ever want your modifications to be
implemented in the authorized version of SWAN (the version on the SWAN web page), you need
to submit these changes to the SWAN team ({\it swan-info-citg@tudelft.nl}).
\newpage
\noindent
The source code is being attended with the following files:
\begin{tabbing}
xxxxxxxxxxxxxxxxxxxxxxxx\= \kill
installation procedures   \>:       INSTALL.README \\
                          \> $\,\,$ Makefile \\
                          \> $\,\,$ Makefile.latex \\
                          \> $\,\,$ macros.inc \\
                          \> $\,\,$ getcmpl \\
                          \> $\,\,$ platform.pl \\
                          \> $\,\,$ switch.pl \\
                          \> $\,\,$ adjlfh.pl \\
run procedures            \>:       SWANRUN.README \\
                          \> $\,\,$ swanrun \\
                          \> $\,\,$ swanrun.bat \\
machinefile for parallel  \> $\,\,$ \\
MPI runs                  \>:       machinefile \\
for concatenation of      \> $\,\,$ \\
multiple hotfiles         \>:       swanhcat.ftn \\
                          \> $\,\,$ hcat.nml \\
edit file                 \>:       swan.edt \\
for conversion of spectra \>:       cvspec1d.for \\
                          \> $\,\,$ cvspec2d.for \\
Matlab scripts for        \> $\,\,$ \\
unstructured grids        \>:       plotunswan.m \\
                          \> $\,\,$ plotgrid.m \\
documentations            \>:       swanuse.tex \\
                          \> $\,\,$ swanuse.pdf \\
                          \> $\,\,$ swanimp.tex \\
                          \> $\,\,$ swanimp.pdf \\
                          \> $\,\,$ swantech.tex \\
                          \> $\,\,$ swantech.pdf \\
                          \> $\,\,$ swanpgr.tex \\
                          \> $\,\,$ swanpgr.pdf \\
                          \> $\,\,$ latexfordummies.tex \\
                          \> $\,\,$ latexfordummies.pdf \\
                          \> $\,\,$ conc.lst \\
                          \> $\,\,$ *.eps, *.ps \\
\end{tabbing}

\noindent
The Latex files (*.tex) can be read and written by any editor and can be compiled with \LaTeXe.
This compilation can also be done automatically; see Section~\ref{sec:makedoc}.
\\[2ex]
\noindent
On the SWAN web page, you also find some test cases with some output files for making a
configuration test of SWAN on your computer. You may compare your results with those in
the provided output files.

\chap{Use of patch files} \label{ch:patch}

Between releases of authorised SWAN versions, it is possible that bug fixes or new
features are published on the SWAN web page. These are provided by patch files that can
be downloaded from the web site. Typically, a patch can be installed over the top of the
existing source code. Patches are indicated by a link to {\bf patchfile}. The names
refer to the current version number supplemented with letter codes. The first will be
coded 'A' (i.e. 40.72.A), the second will be coded 'B', the third will be coded 'C', etc.
The version number in the resulting output files will be updated to 40.72ABC, indicating
the implemented patches.
\\[2ex]
\noindent
To use a patch file, follow the next instructions:
\begin{enumerate}
  \item download the file (right-click the file and choose {\it save link as})
  \item place it in the directory where the source code of SWAN is located
  \item execute {\tt patch -p0 < patchfile}
\end{enumerate}
After applying a patch or patches, you need to recompile the SWAN source code.
\\[2ex]
\noindent
It is important to download the patch and not cut and paste it from the display
of your web browser. The reason for this is that some patches may contain tabs,
and most browsers will not preserve the tabs when they display the file. Copying
and pasting that text will cause the patch to fail because the tabs would not be
found. If you have trouble with patch, you can look at the patch file itself.
\\[2ex]
\noindent
Note to UNIX/Linux users:
the downloaded patch files are MS-DOS ASCII files and contain carriage return (CR) characters.
To convert these files to  UNIX format, use the command {\tt dos2unix}. Alternatively,
execute {\tt cat 40.72.[A-C] | tr -d '$\backslash$r' | patch} that apply the patch files
40.72.A to 40.72.C to the SWAN source code at once after which the conversion is
carried out.
\\[2ex]
\noindent
Note to Windows users: {\tt patch} is a UNIX command. Download the patch program
from the SWAN web site, which is appropriate for Windows operating system (NT/2000/XP).

\chap{Installation of SWAN on your computer} \label{ch:instal}

The portability of the SWAN code between different platforms is guaranteed by the use of standard ANSI
FORTRAN 90. Hence, virtually all Fortran compilers can be used for installing SWAN. See also the manual Programming rules.
\\[2ex]
\noindent
The SWAN code is parallelized, which enables a considerable reduction in the turn-around time for relatively
large CPU-demanding calculations. Two parallelization strategies are available:
\begin{itemize}
  \item The computational kernel of SWAN contains a number of OpenMP compiler directives, so that users can
        optionally run SWAN on a dual core PC or multiprocessor systems.
  \item A message passing modelling is employed based on the Message Passing Interface (MPI) standard that
        enables communication between independent processors. Hence, users can optionally run SWAN on a
        Linux cluster.
\end{itemize}
\noindent
The material on the SWAN web site provides a {\tt Makefile} and two Perl scripts ({\tt platform.pl}
and {\tt switch.pl}) that enables the user to quickly install SWAN on the computer in a
proper manner. For this, the following platforms, operating systems and compilers are supported:
\newpage
\begin{table}[htb]
\begin{tabular}{|l|l|l|}
\hline
{\bf platform} & {\bf OS} & {\bf F90 compiler} \\
\hline
SGI Origin 3000 (Silicon Graphics)  & IRIX        & SGI \\
IBM SP                              & AIX         & IBM \\
Compaq True 64 Alpha (DEC ALFA)     & OSF1        & Compaq \\
Sun SPARC                           & Solaris     & Sun \\
PA-RISC (HP 9000 series 700/800)    & HP-UX v11   & HP \\
IBM Power6 (pSeries 575)            & Linux       & IBM \\
Intel Pentium (32-bit) PC           & Linux       & GNU (g95) \\
Intel Pentium (32-bit) PC           & Linux       & GNU (gfortran) \\
Intel Pentium (32-bit) PC           & Linux       & Intel \\
Intel Pentium (64-bit) PC           & Linux       & Intel \\
Intel Itanium (64-bit) PC           & Linux       & Intel \\
Intel Pentium (64-bit) PC           & Linux       & Portland Group \\
Intel Pentium (32-bit) PC           & Linux       & Lahey \\
Intel Pentium (32-bit) PC           & MS Windows  & Intel \\
Intel Pentium (64-bit) PC           & MS Windows  & Intel \\
Intel Pentium (32-bit) PC           & MS Windows  & Compaq Visual \\
Power Mac G4                        & Mac OS X    & IBM \\
\hline
\end{tabular}
\end{table}

\noindent
If your computer and available compiler is mentioned in the table, you may consult Section
\ref{sec:quick} for a quick installation of SWAN. Otherwise, read Section \ref{sec:manual}
for a detailed description of the manual installation of SWAN.
\\[2ex]
\noindent
Note that for a successful installation, a Perl package must be available on your computer.
In most cases, it is available for Linux and a UNIX operating system. Check it by typing
{\tt perl -v}. Otherwise, you can download a free distribution for Windows called ActivePerl;
see \hl{http://aspn.activestate.com/ASPN/Downloads/ActivePerl/Source}. The Perl version
should be at least 5.0.0 or higher!
\\[2ex]
\noindent
Before installation, the user may first decide how to run the SWAN program. There are three
possibilities:
\begin{itemize}
  \item serial runs,
  \item parallel runs on shared memory systems or
  \item parallel runs on distributed memory machines.
\end{itemize}
For stationary and small-scale computations, it may be sufficient to choose the serial mode, i.e.
one SWAN program running on one processor. However, for relatively large CPU-demanding calculations
(e.g., instationary or nesting ones), two ways of parallelism for reducing the turn-around time are
available:
\begin{itemize}
   \item The SWAN code contains a number of so-called OpenMP directives that enables the
         compiler to generate multithreaded code on a shared memory computer. For this,
         you need a Fortran 90 compiler supporting OpenMP 2.0. The performance is good
         for a restricted number of threads ($< 8$). This type of parallelism can be used e.g.,
         on (symmetric) multiprocessors and PC's with dual core or quad core processors.
   \item If the user want to run SWAN on a relative large number of processors, a message passing
         model is a good alternative. It is based on independent processors which do not share
         any memory but are connected via an interconnection network, such as Beowulf systems
         (cluster of Linux PC's connected via fast Ethernet switches). The implementation is based on
         the Message Passing Interface (MPI) standard (e.g., MPICH2 distribution, freely available
         for several platforms, such as Linux and Windows NT/2000/XP, at
         \hl{http://www-unix.mcs.anl.gov/mpi/mpich}).
         The SWAN code contains a set of generic subroutines that call a
         number of MPI routines, meant for local data exchange, gathering data, global reductions,
         etc. This technique is beneficial for larger simulations only, such that the communication
         times are relatively small compared to the computing times.
\end{itemize}

\section{Automatic and quick installation} \label{sec:quick}

Carry out the following steps for setting up SWAN on your computer.
\begin{enumerate}
  \item An include file containing some machine-dependent macros must be created first.
        This file is called {\tt macros.inc} and can be created by typing
        \begin{verbatim}
            make config
        \end{verbatim}
  \item Now, SWAN can be built for serial or parallel mode, as follows:
        \begin{table}[htb]
           \begin{center}
           \begin{tabular}{|l|l|}
              \hline
              {\bf mode}            & {\bf instruction} \\
              \hline
              serial                & {\tt make ser} \\
              \hline
              parallel, shared      & {\tt make omp} \\
              \hline
              parallel, distributed & {\tt make mpi} \\
              \hline
           \end{tabular}
           \end{center}
        \end{table}
\end{enumerate}
\newpage
\noindent
IMPORTANT NOTES:
\begin{itemize}
  \item To Windows users:
        \begin{itemize}
           \item To execute the above instructions, just open a command prompt.
           \item To build SWAN on Windows platforms by means of a Makefile you need a make program.
                 However, such a program is not available in MS Windows environment. You can download
                 {\tt Nmake} 1.5 for Win32 that is freely available
                 (see http://support.microsoft.com/default.aspx?scid=kb;en-us;Q132084). Run the
                 downloaded file ({\tt Nmake15.exe}) to extract it. Copy both the {\tt NMAKE.EXE} and
                 the {\tt NMAKE.ERR} file to SWAN directory.
           \item This setup does support OpenMP for Windows dual core systems, if Intel Fortran
                 compiler is provided.
           \item This installation currently supports MPICH for
                 Windows NT/2000/XP (Professional); Win9x/ME are not supported.
           \item It is assumed that both the directories {\tt include} and
                 {\tt lib} are resided in\\
                 {\tt C:$\backslash$PROGRAM FILES$\backslash$MPICH$\backslash$SDK}.
                 If not, the file {\tt macros.inc} should be adapted such
                 that they can be found by the Makefile.
        \end{itemize}
  \item One of the commands {\tt make ser}, {\tt make omp} and {\tt make mpi}
        must be preceded by {\tt make config}.
  \item If desirable, you may clean up the generated object files and modules by
        typing {\tt make~clean}. If you want to go back to the original state
        with respect to the source code, i.e. removing everything that has been
        generated by the Makefile, just type {\tt make~clobber}.
  \item If you are unable to install SWAN using the Makefile and Perl scripts for
        whatever reason, see Section \ref{sec:manual} for instructions on manual
        installation.
\end{itemize}

\section{Manual installation} \label{sec:manual}

\subsection{Modifications in the source code}

To compile SWAN on your computer system properly, some subroutines should be adapted first
depending on the operating system, use of compilers and the wish to use MPI for
parallel runs. This can be done by removing the switches started with '!' followed by an
indentifiable prefix in the first 3 or 4 columns of the subroutine. A Perl script called
{\tt switch.pl} is provided in the material that enables the user to quickly select the
switches to be removed. This script can be used as follows:
\begin{verbatim}
perl switch.pl [-dos] [-unix] [-f95] [-mpi] [-cray] [-sgi]
               [-cvis] [-timg] [-impi] *.ftn
\end{verbatim}
where the options are all optionally. The meaning of these options are as follows.
\begin{itemize}
  \item[{\tt -dos}, {\tt -unix}]
  Depending on the operating system, both the TAB and directory separator character must
  have a proper value (see also Chapter \ref{ch:usechan}). This can be done by removing
  the switch !DOS or !UNIX, for Windows and UNIX/Linux platforms, respectively, in the
  subroutines {\tt OCPINI} (in {\tt ocpids.ftn}) and {\tt TXPBLA} (in {\tt swanser.ftn}).
  For other operating system (e.g., Macintosh), you should change the values of the following
  variables manually: {\scriptsize DIRCH1}, {\scriptsize DIRCH2} (in {\tt OCPINI}),
  {\scriptsize TABC} (in {\tt OCPINI}) and {\scriptsize ITABVL} (in {\tt TXPBLA}).
  \item[{\tt -f95}]
  If you have a Fortran 95 compiler or a Fortran 90 compiler that supports Fortran 95
  features, it might be useful to activate the {\tt CPU\_TIME} statement in the subroutines
  {\tt SWTSTA} and {\tt SWTSTO} (in {\tt swanser.ftn}) by removing the switch !F95 meant for
  the detailed timings of several parts of the SWAN calculation. Note that this can be
  obtained with the command TEST by setting {\bf itest=1} in your command file.
  \item[{\tt -mpi}]
  For the proper use of MPI, you must remove the switch !MPI at several places in the file
  {\tt swanparll.ftn}, {\tt swancom1.ftn} and {\tt swmod1.ftn}.
  \item[{\tt -cray}, {\tt -sgi}]
  If you use a Cray or SGI Fortran 90 compiler, the subroutines {\tt OCPINI} (in {\tt ocpids.ftn})
  and {\tt FOR} (in {\tt ocpmix.ftn}) should be adapted by removing the switch !/Cray or !/SGI since,
  these compilers cannot read/write lines longer than 256 characters by default. By means
  of the option {\scriptsize RECL} in the OPEN statement sufficiently long lines can be
  read/write by these compilers.
  \item[{\tt -cvis}]
  The same subroutines {\tt OCPINI} and {\tt FOR} need also to be adapted when the Compaq Visual
  Fortran compiler is used in case of a parallel MPI run. Windows systems have a well-known problem
  of the inability of opening a file by multiple SWAN executables. This can be remedied by using the
  option {\scriptsize SHARED} in the OPEN statement for shared access. For this, just remove the
  switch !CVIS.
  \item[{\tt -timg}]
  If the user want to print the timings (both wall-clock and CPU times in seconds) of different
  processes within SWAN then remove the switch !TIMG. Otherwise, no timings will be keeped up
  and subsequently printed in the {\tt PRINT} file.
  \item[{\tt -impi}]
  Some Fortran compilers do not support {\tt USE MPI} statement and therefore,
  the module {\tt MPI} in {\tt swmod1.ftn} must be included by removing the switch !/impi.
\end{itemize}

\noindent
For example, you work on a Linux cluster where MPI has been installed and use the Intel Fortran
compiler (that can handle Fortran 95 statements), then type the following:
\begin{verbatim}
perl switch.pl -unix -f95 -mpi *.ftn *.ftn90
\end{verbatim}
Note that due to the option {\tt -unix} the extension {\tt ftn} is automatically changed into {\tt f}
and {\tt ftn90} into {\tt f90}.

\subsection{Compiling and linking SWAN source code}

After the necessary modifications are made as described in the previous section, the source code is
ready for compilation. All source code is written in Fortran 90 so you must have a
Fortran 90 compiler in order to compile SWAN. The source code cannot be compiled with a Fortran 77
compiler. If you intended to use MPI for parallel runs, you must use the command {\tt mpif90} instead
of the original compiler command or using the Integrated Development Environment e.g., for Visual
Fortran (see Installation and User's Guide for MPICH).
\\[2ex]
\noindent
The SWAN source code complies with the ANSI Fortran 90 standard, except for a few cases, where
the limit of 19 continuation lines is violated. We are currently not aware of any compiler that cannot
deal with this violation of the ANSI standard.
\\[2ex]
\noindent
When compiling SWAN you should check that the compiler allocates the same amount of memory
for all {\scriptsize INTEGERS}, {\scriptsize REAL} and {\scriptsize LOGICALS}. Usually, for these
variables 4 bytes are allocated, on supercomputers (vector or parallel), however, this sometimes
is 8 bytes. When a compiler allocates 8 bytes for a {\scriptsize REAL} and 4 bytes for an
{\scriptsize INTEGER}, for example, SWAN will not run correctly.
\\[2ex]
\noindent
Furthermore, SWAN can generate binary MATLAB files on request, which are unformatted. Some compilers,
e.g. Compaq Visual Fortran and Intel Fortran version 9.x, measured record length in 4-byte or longword units and
as a consequence, these unformatted files cannot be loaded in MATLAB. Hence, in such as case a
compiler option is needed to request 1-byte units, e.g. for Compaq Visual Fortran this is
{\tt /assume:byterecl} and for Intel Fortran version 9.x this is {\tt -assume byterecl}.
\\[2ex]
\noindent
The modules (in files {\tt swmod1.ftn}, {\tt swmod2.ftn}, {\tt m\_constants.ftn90},
{\tt m\_fileio.ftn90}, {\tt serv\_xnl4v5.ftn90}, {\tt mod\_xnl4v5.ftn90},
{\tt SwanGriddata.ftn90}, {\tt SwanGridobjects.ftn90} and {\tt SwanCompdata.ftn90}) must be compiled first.
Several subroutines use the modules. These subroutines need the compiled versions of the modules before they can be compiled.
Linking should be done without any options nor using shared libraries (e.g. math or NAG). It is
recommended to rename the executable to {\tt swan.exe} after linking.
\\[2ex]
\noindent
Referring to the previous example, compilation and linking may be done as follows:
\begin{verbatim}
mpif90 swmod1.f swmod2.f m_constants.f90 m_fileio.f90 serv_xnl4v5.f90
       mod_xnl4v5.f90 SwanGriddata.f90 SwanGridobjects.f90 SwanCompdata.f90
       ocp*.f swan*.f Swan*.f90 -o swan.exe
\end{verbatim}

\newpage
\section{Make SWAN documentation} \label{sec:makedoc}

SWAN comes with 4 detailed documents which are provided as downloadable PDF files as well as browsable web pages:
\begin{itemize}
  \item The \underline{User Manual} describes the complete input and usage of the SWAN package.
  \item The \underline{Implementation Manual} explains the installation procedure of SWAN on a single- or multi-processor machine
        with shared or distributed memory.
  \item The \underline{Programming rules} is meant for programmers who want to develop SWAN.
  \item The \underline{Technical documentation} discusses the mathematical details and the discretizations that are used in the SWAN program.
\end{itemize}
These documents are written in \LaTeX~format. If you are new to \LaTeX, we recommend to read first the manual \underline{\LaTeX~for dummies}
that is available in the SWAN material.
\\[2ex]
\noindent
The PDF files are very easy to generate by just typing
\begin{verbatim}
    make doc
\end{verbatim}

\chap{User dependent changes and the file {\tt swaninit}} \label{ch:usechan}

SWAN allows you to customize the input and the output to the wishes of your
department, company or institute. This can be done by changing the settings
in the initialisation file {\tt swaninit}, which is created during the first time
SWAN is executed on your computer system. The changes in {\tt swaninit} only affect
the runs executed in the directory that contains that file.
\\[2ex]
\noindent
A typical initialisation file {\tt swaninit} may look like:
\begin{verbatim}
    4                               version of initialisation file
Delft University of Technology      name of institute
    3                               command file ref. number
INPUT                               command file name
    4                               print file ref. number
PRINT                               print file name
    4                               test file ref. number
                                    test file name
    6                               screen ref. number
   99                               highest file ref. number
$                                   comment identifier
[TAB]                               TAB character
\                                   dir sep char in input file
/                                   dir sep char replacing previous one
    1                               default time coding option
  100                               speed of processor   1
  100                               speed of processor   2
  100                               speed of processor   3
  100                               speed of processor   4
\end{verbatim}
\newpage
\noindent
Explanation:
\begin{itemize}
   \item The version number of the initialisation file is included in the file so that SWAN can
         verify whether the file it reads is a valid initialisation file. The current version is 4.
   \item The initialisation file provides a character string containing the name of the institute
         that may carry out the computations or modifying the source code. You may assign it to
         the name of your institute instead of 'DELFT UNIVERSITY OF TECHNOLOGY', which is the
         present value.
   \item The standard input file and standard print file are usually named {\tt INPUT} and
         {\tt PRINT}, respectively. You may rename these files, if appropriate.
   \item The unit reference numbers for the input and print files are set to 3 and 4, respectively.
         If necessary, you can change these numbers into the standard input and output unit numbers
         for your installation. Another unit reference number is foreseen for output to screen and
         it set to 6. There is also a unit number for a separate test print file. In the
         version that you downloaded from our web page, this is equal to that of the print file so
         that test output will appear on the same file as the standard print output.
   \item The comment identifier to be used in the command file is usually '\$', but on some computer
         system this may be inappropriate because a line beginning with '\$' is interpreted as a
         command for the corresponding operating system (e.g., VAX systems). If necessary, change
         to '!'.
   \item To insert [TAB] in the initialisation file, just use the TAB key on your keyboard.
   \item Depending on the operating system, the first directory separation character in {\tt swaninit},
         as used in the input file, may be replaced by the second one, if appropriate.
   \item Date and time can be read and written according to various options. The following options
         are available:
         \begin{enumerate}
            \item 19870530.153000 (ISO-notation)
            \item 30-May-87 15:30:00
            \item 05/30/87  15:30:00
            \item 15:30:00
            \item 87/05/30  15:30:00
            \item 8705301530 (WAM-equivalence)
         \end{enumerate}
         Note that the ISO-notation has no millenium problem, therefore the ISO-notation is
         recommended. In case of other options, the range of valid dates is in between January 1,
         1911 and December 31, 2010 (both inclusive).
   \item In case of a parallel MPI run at the machine having a number of independent processors,
         it is important to assign subdomains representing appropriate amounts of work to each
         processor. Usually, this refers to an equal number of grid points per subdomain. However,
         if the computer has processors which are not all equally fast (a so-called heterogeneous
         machine), then the sizes of the subdomains depend on the speed of the processors.
         Faster processors should deal with more grid points than slower ones. Therefore, if
         necessary, a list of non-default processor speeds is provided. The given speeds are in
         \% of default = 100\%. As an illustrating example, we have two PC's connected via an
         Ethernet switch of which the first one is 1.5 times faster than the second one. The list
         would be
         \begin{verbatim}
         150    speed of processor 1
         100    speed of processor 2
         \end{verbatim}
         Based on this list, SWAN will automatically distribute the total number of active grid points
         over two subdomains in an appropriate manner. Referring to the above example, with 1000 active
         points, the first and second subdomains will contain 600 and 400 grid points, respectively.
\end{itemize}

\chap{Usage of SWAN executable} \label{ch:usage}

To help you in editing an command file for SWAN input, the file {\tt swan.edt} is provided.
\\[2ex]
\noindent
Two run procedures are provided among the source code, one for the Windows platform, called
{\tt swanrun.bat}, and one for the UNIX/Linux platform, called {\tt swanrun}. Basically, the
run procedure carries out the following actions:
\begin{itemize}
  \item Copy the command file with extension {\tt swn} to {\tt INPUT} (assuming {\tt INPUT} is
        the standard file name for command input, see Chapter \ref{ch:usechan}).
  \item Run SWAN.
  \item Copy the file {\tt PRINT} (assuming {\tt PRINT} is the standard file name for print
        output, see Chapter \ref{ch:usechan}) to a file which name equals the command file
        with extension {\tt prt}.
\end{itemize}
On other operating system a similar procedure can be followed. For parallel MPI runs, the
program {\tt mpirun} is needed and is provided in the MPICH distribution.
\\[2ex]
\noindent
Before calling the run procedure, the environment variable {\small \tt PATH} need to be adapted by
including the pathname of the directory where {\tt swan.exe} can be found. In case of Windows,
this pathname can be specified through the category {\it System} of {\it Control Panel} (on the
{\it Advanced} tab, click {\it Environment Variables}) or by adding it in the {\tt AUTOEXEC.BAT}
file. In case of UNIX or Linux running the bash shell (sh or ksh), the environment variable
{\small \tt PATH} may be changed as follows:
\begin{verbatim}
export PATH=${PATH}:/usr/local/swan
\end{verbatim}
if {\tt /usr/local/swan} is the directory where the executable {\tt swan.exe} is resided.
In case of the C shell (csh), use the following command:
\begin{verbatim}
setenv PATH ${PATH}:/usr/local/swan
\end{verbatim}

\noindent
If appropriate, you also need to add the directory path where the {\tt bin} directory of MPICH
is resided to {\small \tt PATH} to have access to the command {\tt mpirun}.
\\[2ex]
\noindent
You may also specify the number of threads to be used during execution of the multithreaded
implementation of SWAN on multiprocessor systems. The environment variable for this is
{\small \tt OMP\_NUM\_THREADS} and can be set like
\begin{verbatim}
export OMP_NUM_THREADS=4
\end{verbatim}
or
\begin{verbatim}
setenv OMP_NUM_THREADS 4
\end{verbatim}
or, in case of Windows,
\begin{verbatim}
OMP_NUM_THREADS = 4
\end{verbatim}
When dynamic adjustment of the number of threads is enabled, the value given in\\
{\small \tt OMP\_NUM\_THREADS} represents the maximum number of threads allowed.
\\[2ex]
\noindent
The provided run procedures enable the user to properly and easily run SWAN both serial as
well as parallel (MPI or OpenMP). Note that for parallel MPI runs, the executable {\tt swan.exe}
should be accessible by copying it to all the multiple machines or by placing it in a shared
directory. When running the SWAN program, the user must specify the name of the command file.
However, it is assumed that the extension of this file is {\tt swn}. Note that contrary to
UNIX/Linux, Windows does not distinguish between lowercase and uppercase characters in filenames.
Next, the user may also indicate whether the run is serial or parallel. In case of Windows, use
the run procedure {\tt swanrun.bat} from a command prompt:
\begin{verbatim}
swanrun filename [nprocs]
\end{verbatim}
where {\tt filename} is the name of your command file without extension (assuming it is {\tt swn})
and {\tt nprocs} indicates how many processes need to be launched for a parallel MPI run (do not
type the brackets; they just indicate that {\tt nprocs} is optional). By default, $nprocs = 1$.
You may also run on a dual/quad core computer; do not set {\tt nprocs}.
\\[2ex]
\noindent
The command line for the UNIX script {\tt swanrun} is as follows:
\begin{verbatim}
./swanrun -input filename [-omp n | -mpi n]
\end{verbatim}
where {\tt filename} is the name of your command file without extension. Note that the script
{\tt swanrun} need to be made executable first, as follows:
\begin{verbatim}
chmod +rx ./swanrun
\end{verbatim}
The parameter {\tt -omp~n}
specifies a parallel run on $n$ cores using OpenMP. Note that the UNIX script will set
{\small \tt OMP\_NUM\_THREADS} to $n$. The parameter {\tt -mpi~n} specifies a
parallel run on $n$ processors using MPI. The parameter {\tt -input} is obliged, whereas the
parameters {\tt -omp~n} and {\tt -mpi~n} can be omitted (default: $n=1$). To redirect screen
output to a file, use the sign $>$. Use an ampersand to run SWAN in the background. An
example:
\begin{verbatim}
./swanrun -input f31har01 -omp 4 > swanout &
\end{verbatim}

\noindent
For a parallel MPI run, you may also need a {\tt machinefile} that contains the names of the
nodes in your parallel environment. Put one node per line in the file. Lines starting with
the \# character are comment lines. You can specify a number after the node name to indicate
how many cores to launch on the node. This is useful e.g., for multi-core processors. The run
procedure will cycle through this list until all the requested processes are launched. Example
of such a file may look like:
\begin{verbatim}
# here, eight processes will be launched
node1
node2:2
node4
node7:4
\end{verbatim}
Note that for Windows platforms, a space should be used instead of a colon as the separation
character in the {\tt machinefile}.
\\[2ex]
\noindent
SWAN will generate a number of output files:
\begin{itemize}
   \item A print file with the name {\tt PRINT} that can be renamed by the user with a batch (DOS) or
         script (UNIX) file, e.g. with the provided run procedures. For parallel MPI runs, however,
         a sequence of {\tt PRINT} files will be generated ({\tt PRINT-001}, {\tt PRINT-002}, etc.)
         depending on the number of processors. The print file(s) contain(s) the echo of the input,
         information concerning the iteration process, possible errors, timings, etc.
   \item Numerical output (such as table, spectra and block output) appearing in files with user
         provided names.
   \item A file called {\tt Errfile} (or renamed by the run procedures as well as more than one file in
         case of parallel MPI runs) containing the error messages is created only when SWAN produces
         error messages. Existence of this file is an indication to study the results with more care.
   \item A file called {\tt ERRPTS} (or renamed by the run procedures as well as more than one file in
         case of parallel MPI runs) containing the grid-points, where specific errors occured during
         the calculation, such as non-convergence of an iterative matrix-solver. Existence of this file
         is an indication to study the spectrum in that grid-point with more care.
\end{itemize}
If indicated by the user, a single or multiple hotfiles will be generated depending on the number of
processors, i.e. the number of hotfiles equals the number of processors (see the User Manual). Restarting
a (parallel MPI) run can be either from a single (concatenated) hotfile or from multiple hotfiles.
In the latter case, the number of processors must be equal to the number of generated hotfiles. If
appropriate, the single hotfile can be created from a set of multiple hotfiles using the program
{\tt hcat.exe} as available from SWAN version 40.51A. This executable is generated from the Fortran
program {\tt swanhcat.ftn}. A self-contained input file {\tt hcat.nml} is provided. This file contains,
e.g. the (basis) name of the hotfile. To concatenate multiple hotfiles into single hotfile just execute
{\tt hcat.exe}.

\chap{Testing the system} \label{ch:test}

The SWAN system consists of one executable file ({\tt swan.exe}), a command file ({\tt swan.edt}) and
a run procedure ({\tt swanrun.bat} or {\tt swanrun}).
Depending on your system, you may use 32-bit or 64-bit executable.
These executables for Windows NT/2000/XP can be obtained from the SWAN web site.
The input and output to a number of test problems
is provided on the SWAN web page. The files with extension {\tt swn} are the command files for these
tests; the files with extension {\tt bot} are the bottom files for these tests, etc. This input can be
used to make a configuration test of SWAN on your computer. Compare the results with those in the
provided output files. Note that the results need not to be identical up to the last digit.
\\[2ex]
\noindent
To run the SWAN program for the test cases, at least 50 MBytes of free internal memory is recommended.
For more realistic cases 100 to 500 MBytes may be needed, whereas for more simple stationary or 1D
cases significant less memory is needed (less than 5 MBytes for 1D cases).

\end{document}
